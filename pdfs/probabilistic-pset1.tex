\documentclass[12pt]{article}

\usepackage{amsmath}
\usepackage{amssymb}
\usepackage{xcolor}
\usepackage[bmargin=1in]{geometry}
\usepackage[inline]{asymptote}
\usepackage{comment}

\title{Probabilistic Method Pset 1}
\author{EGMOTC 2023 - Rohan}
\date{\today}
\setlength{\parindent}{0pt}

\begin{document}

\maketitle

\newcommand{\localtextbulletone}{\textcolor{black}{\raisebox{.45ex}{\rule{.6ex}{.6ex}}}}
\renewcommand{\labelitemi}{\localtextbulletone}

\section*{Problems}
\vspace{1cm}
\thispagestyle{empty}

\textbf{Remark.} \textit{* marked problems are considered harder.\\ ** marked problems are strictly optional for the ones feeling extremely curious\\}


\textbf{Problem.} Suppose you have the whole numbers number line, labelled $0,1,2,\ldots$ and a drunk person say, Aditi, starts at $x=7$ and every minute with equal probability goes either left($-1$) or right($+1$). What's the probability that:
    \begin{enumerate}
        \item She reaches Ananya who's standing at $x=10$ first or Sunaina who's standing at $x=0$ first?
        \item What's the expected amount of time before she reaches atleast one of them?
        \item (*) Can you answer the same questions for general values instead of $0,7$ and $10$? What if she goes right with probability $p$ and left with probability $1-p$?
        \item (**) Can we ask similar questions in higher dimensions? What if they are playing the same game on the entire integer plane?
    \end{enumerate}


\end{document}