\documentclass[12pt]{article}

\usepackage{amsmath}
\usepackage{amssymb}
\usepackage{xcolor}
\usepackage[bmargin=1in]{geometry}
\usepackage[inline]{asymptote}
\usepackage{comment}

\title{Probabilistic Method Pset 1}
\author{EGMOTC 2023 - Rohan}
\date{\today}
\setlength{\parindent}{0pt}

\begin{document}

\maketitle

\newcommand{\localtextbulletone}{\textcolor{black}{\raisebox{.45ex}{\rule{.6ex}{.6ex}}}}
\renewcommand{\labelitemi}{\localtextbulletone}

\section*{Problems}
\vspace{1cm}
\thispagestyle{empty}

\textbf{Remark.} \textit{* marked problems are considered harder.\\ ** marked problems are strictly optional for the ones feeling extremely curious about this particular setup.\\}
\textbf{Remark.} Try to do the first two parts atleast and submit whatever progress you get on the last two parts.\\

\textbf{Problem.} Suppose you have the integer number line, labelled $\ldots,-2,-1,0,1,2,\ldots$ and a drunk Aditi starts at $x=7$ and every minute with equal probability goes either left($-1$) or right($+1$).
    \begin{enumerate}
        \item Ananya is standing at $x=0$ and Sunaina is at $x=10$. As soon as she reaches either of them, she stops moving and starts talking to them. What's the probability that she reaches Sunaina?
        \item What's the expected amount of time before she reaches atleast one of them?
        \item (*) Can you answer the same questions for general values instead of $0,7$ and $10$? What if she goes right with probability $p$ and left with probability $1-p$?
        \item (**) Now, assume we are on the integer lattice ($\mathbb{Z}\times \mathbb{Z}$) and Sahil is standing at the origin with an iced tea stand for her and she is at $(5,2)$ then what is the probability that she gets an infinite number of iced teas if she moves in each of the 4 directions with equal probability? What if she was in a \textit{higher} dimensional space?  
    \end{enumerate}


\end{document}