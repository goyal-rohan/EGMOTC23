\documentclass[12pt]{article}

\usepackage{amsmath}
\usepackage{amssymb}
\usepackage{xcolor}
\usepackage[bmargin=1in]{geometry}
\usepackage[inline]{asymptote}
\usepackage{comment}

\title{Probabilistic Method Pset 2}
\author{EGMOTC 2023 - Rohan}
\date{\today}
\setlength{\parindent}{0pt}

\begin{document}

\maketitle

\newcommand{\localtextbulletone}{\textcolor{black}{\raisebox{.45ex}{\rule{.6ex}{.6ex}}}}
\renewcommand{\labelitemi}{\localtextbulletone}

\section*{Problems}
\vspace{1cm}
\thispagestyle{empty}

\textbf{Remark.} \textit{* marked problems are considered harder.\\ ** marked problems are strictly optional for the ones feeling extremely curious}

\begin{enumerate}
    \item (MP4G 2022) Across the face of a rectangular post-it note, you idly draw lines that are parallel to its edges. Each time you draw a line, there is a $50\%$ chance it'll be in each direction and you never draw over an existing line or the edge of the post-it note. After a few minutes, you notice that you've drawn 20 lines. What is the expected number of rectangles that the post-it note will be partitioned into?
    \item (Folklore) Let $v_1,v_2,\ldots v_n\in \mathbb{R}^n$, all $|v_i|=1$ where $|x|$ refers to the Euclidean distance of $x$ from the origin.  Then there exist $\epsilon_1, \epsilon_2, \ldots \epsilon_n = \pm 1$ such that \[|\epsilon_1v_1+\epsilon_2v_2+\cdots \epsilon_nv_n|\le \sqrt{n}\] and also there exist $\epsilon_1, \epsilon_2, \ldots \epsilon_n = \pm 1$ such that \[|\epsilon_1v_1+\epsilon_2v_2+\cdots \epsilon_nv_n|\ge \sqrt{n}\] 
    \item (*) Suppose $p>n>10m^2$, with $p$ prime, and let $0<a_1<a_2<\cdots a_m<p$ be integers. Prove that there is an integer $0<x<p$ for which the $m$ numbers \[(xa_i \pmod{p})\pmod n\] are pairwise distinct.
\end{enumerate}

\end{document}