\documentclass[12pt]{article}

\usepackage{amsmath}
\usepackage{amssymb}
\usepackage{xcolor}
\usepackage[bmargin=1in]{geometry}
\usepackage[inline]{asymptote}
\usepackage{comment}

\title{TST Mock 1}
\author{EGMOTC 2023 - Rohan}
\date{\today}
\setlength{\parindent}{0pt}

\begin{document}

\maketitle

\newcommand{\localtextbulletone}{\textcolor{black}{\raisebox{.45ex}{\rule{.6ex}{.6ex}}}}
\renewcommand{\labelitemi}{\localtextbulletone}

\section*{Problems}
\vspace{1cm}
\thispagestyle{empty}

\paragraph{\textbf{Problem 1.}} A point $P$ lies in $\triangle ABC$. The lines $BP,CP$ meet $AC,AB$ at $Q,R$ respectively. Given that $AR=RB=CP, CQ=PQ$, find $\angle BRC$. %Japan 2003/1

\paragraph{\textbf{Problem 2.}} The two cats Fitz and Will play the following game. On a blackboard is written the expression
\[ 
  x^{100} + {\square} x^{99} + {\square} x^{98} + {\square} x^{97} + \dots + {\square } x^2 + {\square} x +1. 
\]Both cats take alternate turns replacing one $\square$ with a $0$ or $1$, with Fitz going first, until (after 99 turns) all the blanks have been filled. If the resulting polynomial obtained has a real root, then Will wins, otherwise Fitz wins. Determine, with proof, which player has a winning strategy.

\paragraph{\textbf{Problem 3.}} We call a natural number $n$ honourable, if when a single corner cell is removed from an $n \times n$ grid, there are an odd number of ways of tiling the remaining cells using L-trominoes. Prove that a number is honourable if and only if it is a power of $2$

\end{document}