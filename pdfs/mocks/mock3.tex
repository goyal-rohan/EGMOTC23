\documentclass[12pt]{article}

\usepackage{amsmath}
\usepackage{amssymb}
\usepackage{xcolor}
\usepackage[bmargin=-1in]{geometry}
\usepackage[inline]{asymptote}
\usepackage{comment}

\title{TST Mock 3}
\author{EGMOTC 2023 - Rohan}
\date{\today}
\setlength{\parindent}{0pt}

\begin{document}

\maketitle

\newcommand{\localtextbulletone}{\textcolor{black}{\raisebox{.45ex}{\rule{.6ex}{.6ex}}}}
\renewcommand{\labelitemi}{\localtextbulletone}

\section*{Problems}
\vspace{0cm}
\thispagestyle{empty}

\paragraph{\textbf{Problem 1.}} Read the following conversation between Rachel and Randy.\\ %Canada 2001

\textit{Randy}: Hi Rachel, that's an interesting quadratic equation you have written down. What are its roots?\\
\textit{Rachel}: The roots are two positive integers. One of the roots is my age, and the other root is the age of my younger brother, Jimmy.\\
\textit{Randy}: That is very neat! Let me see if I can figure out how old you and Jimmy are. That shouldn't be too difficult since all of your coefficients are integers. By the way, I notice that the sum of the three coefficients is a prime number.\\
\textit{Rachel}: Interesting. Now figure out how old I am.\\
\textit{Randy}: Instead, I will guess your age and substitute it for $x$ in your quadratic equation $\dots$ darn, that gives me $-55$, and not $0$.\\
\textit{Rachel}: Oh, leave me alone!

\begin{enumerate}
    \item Prove that Jimmy is two years old.
    \item Determine Rachel's age.
\end{enumerate}

\paragraph{\textbf{Problem 2.}} A positive integer $a$ is selected, and some positive integers are written on a board. Alice and Bob play the following game. On Alice's turn, she must replace some integer $n$ on the board with $n+a$, and on Bob's turn he must replace some even integer $n$ on the board with $n/2$. Alice goes first and they alternate turns. If on his turn Bob has no valid moves, the game ends.\\

After analyzing the integers on the board, Bob realizes that, regardless of what moves Alice makes, he will be able to force the game to end eventually. Show that, in fact, for this value of $a$ and these integers on the board, the game is guaranteed to end regardless of Alice's or Bob's moves.

\paragraph{\textbf{Problem 3.}} The incircle $\omega$ of a triangle $ABC$ touches the sides $AC$ and $BA$ at $E$ and $F$ respectively. $N$ is the midpoint of arc $BAC$ and $P$ is the foot of altitude from the midpoint of $BC$ onto $EF$. Prove that the line $NP$ passes through the centre of $\omega$.

\end{document}