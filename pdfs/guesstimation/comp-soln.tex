\documentclass[12pt]{article}

\usepackage{amsthm}
\usepackage{amsmath}
\usepackage{amssymb}
\usepackage{xcolor}
\usepackage[bmargin=1in]{geometry}
\usepackage[inline]{asymptote}
\usepackage{comment}
\newenvironment{solution}
{\paragraph{Solution.}}
{\qed\eject}
\usepackage{hyperref}
\newcommand*{\EE}{\mathbb{E}}
\newcommand*{\PP}{\mathbb{P}}
\newcommand*{\NN}{\mathbb{N}}
\newcommand{\hyt}[2]{\hypertarget{#1 Claim #2}{\paragraph{Claim #2}}}
\newcommand{\hyl}[2]{Same as \hyperlink{#1 Claim #2}{Claim #2 in Solution #1}}
\newcommand{\hyr}[2]{\hyperlink{#1 Claim #2}{Claim #2}}
\DeclareMathOperator{\pow}{pow}
\newcommand{\nnt}{90^{\circ}}
\newcommand{\es}{\\[12pt]}
\usepackage{csquotes}
\usepackage{pythonhighlight}

\title{Size and Guesstimation: Computational}
\author{EGMOTC 2023 - Rohan}
\date{\today}
\setlength{\parindent}{0pt}

\begin{document}

\maketitle

\newcommand{\localtextbulletone}{\textcolor{black}{\raisebox{.45ex}{\rule{.6ex}{.6ex}}}}
\renewcommand{\labelitemi}{\localtextbulletone}

\section*{Problems}
\vspace{1cm}
\thispagestyle{empty}


\textbf{Problem 1. (Newton Iteration)} 
\begin{itemize}
    \item Find $\sqrt{2023}$ upto $20$ decimal places (without using the $\sqrt{\cdot{}}$ operation). You are free to write code for this or use a calculator.\footnote{4 iterations of Newton's algorithm are definitely sufficient, 3 iterations might be enough too. Can you argue that 4 iterations are sufficient?}
    \item Find the first 10 digits of $\pi$. \footnote{$sin \pi=0$ and you can use Newton Iteration}
\end{itemize}

\textbf{Problem 2. (Big-O)} Order the function by their sizes as $n\mapsto \infty$?
\begin{itemize}
    \item $f(x)=2023\log(n)^{2023}$, $g(x)=\log(\log(F_n^{F_n^2}+3^{3^n}))$\footnote{$F_n$ is the $n$th Fibonacci term}, $h(x)=1.001^n$
    \item $f(n)=3f\left(\lfloor{} n/2\rfloor{}\right)+2023n$ with $f(1)=1$, $g(n)=1.01g(n-1)-g(n-2)$ with $g(0)=0$ and $g(1)=1$
    \item $f(n)=2023^{2023^{2023n}}$, $g(n)=2^{2^{2^{2n}}}$, and $h(n)=1.01^{1.01^{1.01^{1.01^{1.01n}}}}$  
\end{itemize}  

\textbf{Problem 3. (Some contest problems)}
\begin{itemize}
    \item Compute $\left\lceil\displaystyle\sum_{k=2023}^{\infty}\frac{2024!-2023!}{k!}\right\rceil$
    \item For any natural number $n$, expressed in base 10 , let $s(n)$ denote the sum of all its digits. Find all natural numbers $m$ and $n$ such that $m<n$ and
    \[(s(n))^2=m \text { and }(s(m))^2=n\]
\end{itemize} 

\eject

\section*{Solutions}

\subsection*{Problem 1 (Newton Iteration)} \begin{itemize}
    \item Find $\sqrt{2023}$ upto $20$ decimal places (without using the $\sqrt{\cdot{}}$ operation). You are free to write code for this or use a calculator.
    \item Find the first 10 digits of $\pi$.
\end{itemize}

\begin{solution}
    I have written a short python program for both the approximations. You can check the same using a high-precision calculator too.
    \begin{lstlisting}[language=python]
        k=float(45)
        for i in range(4):
            k=(k+(2023/k))/2
        print(k)
    \end{lstlisting}
    The answer outputted is: $44.97777228809804$. This is because my system can only handle that many bits. So, I put this into a calculator for one more iteration and got:
    \[44.977772288098040038527467811867427253575509885245374316455309088\]
    Wolfram gives \[\sqrt{2023}=44.977772288098040038527467811867427237074406112401653066261685806\] You can see that the ouput we get is indeed correct upto $34$ points after the decimal in just $5$ steps!

    Now, moving onto $\pi$. We observe that $sin(\pi)=0$. So, we use our approximation as $\pi_0=3$ and \[\pi_{i+1}=\pi_i-\frac{\sin(\pi_i)}{\cos(\pi_i)}\]

    Writing the code for this:
    \begin{lstlisting}[language=python]
        import math
        pi=float(3)
        for i in range(4):
            pi=pi-(math.sin(pi)/math.cos(pi))
        print(pi)
    \end{lstlisting}
    My system outputs: $3.141592653589793$. This is again due to my systme not handling better. I applied the approximation again using a better calculator and got:
    \[3.1415926535897932384626433832795028841971693993796258352543\]
    Wolfram gives the value of $\pi$ as:
    \[3.1415926535897932384626433832795028841971693993751058209749445923\]
    This is correct for $47$ digits after the decimal in just $5$ steps!!
\end{solution}

\subsection*{Problem 2.}

\paragraph*{\textbf{Problem 2.} (Big-O)} Order the function by their sizes as $n\mapsto \infty$?
\begin{itemize}
    \item $f(x)=2023\log(n)^{2023}$, $g(x)=\log(\log(F_n^{F_n^2}+3^{3^n}))$\footnote{$F_n$ is the $n$th Fibonacci term}, $h(x)=1.001^n$
    \item $f(n)=3f\left(\lfloor{} n/2\rfloor{}\right)+2023n$ with $f(1)=1$, $g(n)=1.01g(n-1)-g(n-2)$ with $g(0)=0$ and $g(1)=1$
    \item $f(n)=2023^{2023^{2023n}}$, $g(n)=2^{2^{2^{2n}}}$, and $h(n)=1.01^{1.01^{1.01^{1.01^{1.01n}}}}$  
\end{itemize} 

\begin{solution}
    \begin{itemize}
        \item $f(x)=O(\log(x)^{2023})$, $g(x)=O\left(\left(\log\log F_n^{F_n^2}\right)+n\right)=O(n+\log F_n)=O(n)$, $h(x)=O(1.001^n)$. Thus, the first is polylogarithmic, second is linear and third is exponential. Thus, for all large enough $n$, we have $f(n)<g(n)<h(n)$
        \item $f(n)=3f(n/2)+O(n)\implies f(n)=O(n^{3/2})$ by the master theorem. $g(n)=1.01g(n-1)-g(n-2)$ so the answer is of the form $C_1\alpha^n+C_2\beta^n$ where $\alpha$ and $\beta$ are the two roots of $x^{2}=1.01x-1$. Now, both the roots are complex conjugates and have modulus $1$. So, $|g(n)|$ is always bounded!.  
        \item We take log twice for all function. The functions become $\log\log f(n)=O(n)$, $\log\log g(n)=O(4^n)$ and finally $\log\log h(n)=O(1.01^{1.01^{1.01n}})$. Now, to make the comparison between $g(n)$ and $h(n)$ even more clear, let's take log twice more. Now, $\log \log \log \log f(n)=O(\log\log n)$, $\log \log \log \log g(n)=O(\log n)$, and $\log \log \log \log h(n)=O(n)$ 
    \end{itemize}
\end{solution}

\subsection*{Problem 3}
\paragraph*{\textbf{Problem 3.} (Contest problems)} 
\begin{itemize}
    \item Compute $\left\lceil\displaystyle\sum_{k=2023}^{\infty}\frac{2024!-2023!}{k!}\right\rceil$
    \item For any natural number $n$, expressed in base 10 , let $s(n)$ denote the sum of all its digits. Find all natural numbers $m$ and $n$ such that $m<n$ and
    \[(s(n))^2=m \text { and }(s(m))^2=n\]
\end{itemize} 

\begin{solution}
    \begin{itemize}
        \item We write the first part simply as \[2024+\left\lceil{}\sum_{k=2024}^{\infty}\frac{2024!}{(k+1)!}-\frac{2023!}{k!}\right\rceil{}\] Now, for all $k\ge 2024$, we have $\frac{2024!}{(k+1)!}<\frac{2023!}{k!}$ as we have $2024<k+1$.\\
        Thus, the part inside the $\lceil \rceil$ is $<0$. Thus, the value is $\le 2024$. Now, we also have using $k=1$ that the sum is $2023+\left\lceil\displaystyle\sum_{k=2024}\frac{2024!-2023!}{k!}\right\rceil\ge 2024$.

        Thus, the final value is simply $2024$.\qed
        \item This problem is from RMO 2023:
            We first observe that if $n$ has $k$ digits, we get that $s(m)^2=n$ and $m$ also has at most $k$ digits. Thus, $10^{k-1}\le (9k)^2$. This is already false for $n=5$. Thus, $n$ has at most $4$ digits but then $s(m)\le 36\implies n\le 1296$.
            
            Then, \[s(m)\le 27\implies n\le 729\implies s(m)\le 24\implies n\le 576\]\[\implies s(m)\le 22\implies n\le 484\implies s(m)\le 21\implies n\le 441\]
            Now, since $n=441$ doesn't work, we get $n\le 400$ as it is also a perfect square.\\

            At this point, we can simply check all $n$! Possibilities of $n$ are $\{1^2, 2^2,\cdots 20^2\}$ but $s(m)\equiv m\not\equiv 2\pmod{3}$ since $m$ is a square.\\
            
            Thus, $n\in \{1^2, 3^3, 4^2, 6^2, 7^2, 9^2, 10^2, 12^2, 13^2, 15^2, 16^2, 18^2, 19^2\}$. Checking tells us that $n=256$ gives $m=169$ which indeed works!
    \end{itemize}
\end{solution}

\end{document}