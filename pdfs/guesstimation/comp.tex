\documentclass[12pt]{article}

\usepackage{amsmath}
\usepackage{amssymb}
\usepackage{xcolor}
\usepackage[bmargin=1in]{geometry}
\usepackage[inline]{asymptote}
\usepackage{comment}

\title{Size and Guesstimation: Computational}
\author{EGMOTC 2023 - Rohan}
\date{\today}
\setlength{\parindent}{0pt}

\begin{document}

\maketitle

\newcommand{\localtextbulletone}{\textcolor{black}{\raisebox{.45ex}{\rule{.6ex}{.6ex}}}}
\renewcommand{\labelitemi}{\localtextbulletone}

\section*{Problems}
\vspace{1cm}
\thispagestyle{empty}


\textbf{Problem 1. (Newton Iteration)} 
\begin{itemize}
    \item Find $\sqrt{2023}$ upto $20$ decimal places (without using the $\sqrt{\cdot{}}$ operation). You are free to write code for this or use a calculator.\footnote{4 iterations of Newton's algorithm are definitely sufficient, 3 iterations might be enough too. Can you argue that 4 iterations are sufficient?}
    \item Find the first 10 digits of $\pi$. \footnote{$sin \pi=0$ and you can use Newton Iteration)}
\end{itemize}

\textbf{Problem 2. (Big-O)} Order the function by their sizes as $n\mapsto \infty$?
\begin{itemize}
    \item $f(x)=2023\log(n)^{2023}$, $g(x)=\log(\log(F_n^{F_n^2}+3^{3^n}))$\footnote{$F_n$ is the $n$th Fibonacci term}, $h(x)=1.001^n$
    \item $f(n)=3f\left(\lfloor{} n/2\rfloor{}\right)+2023n$ with $f(1)=1$, $g(n)=1.01g(n-1)-g(n-2)$ with $g(0)=0$ and $g(1)=1$
    \item $f(n)=2023^{2023^{2023n}}$, $g(n)=2^{2^{2^{2n}}}$, and $h(n)=1.01^{1.01^{1.01^{1.01^{1.01n}}}}$  
\end{itemize}  

\textbf{Problem 3. (Some contest problems)}
\begin{itemize}
    \item Compute $\left\lceil\displaystyle\sum_{k=2023}^{\infty}\frac{2024!-2023!}{k!}\right\rceil$
    \item For any natural number $n$, expressed in base 10 , let $s(n)$ denote the sum of all its digits. Find all natural numbers $m$ and $n$ such that $m<n$ and
    \[(s(n))^2=m \text { and }(s(m))^2=n\]
\end{itemize} 

\end{document}