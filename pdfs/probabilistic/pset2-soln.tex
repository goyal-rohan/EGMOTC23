\documentclass[12pt]{article}

\usepackage{amsthm}
\usepackage{amsmath}
\usepackage{amssymb}
\usepackage{xcolor}
\usepackage[bmargin=1in]{geometry}
\usepackage[inline]{asymptote}
\usepackage{comment}
\newenvironment{solution}
{\paragraph{Solution.}}
{\qed\eject}
\newcommand*{\EE}{\mathbb{E}}
\newcommand*{\PP}{\mathbb{P}}


\title{Probabilistic Method Pset 2}
\author{EGMOTC 2023 - Rohan}
\date{\today}
\setlength{\parindent}{0pt}

\begin{document}

\maketitle

\newcommand{\localtextbulletone}{\textcolor{black}{\raisebox{.45ex}{\rule{.6ex}{.6ex}}}}
\renewcommand{\labelitemi}{\localtextbulletone}

\section*{Problems}
\vspace{1cm}
\thispagestyle{empty}

\textbf{Remark.} \textit{* marked problems are considered harder.\\ }
\textbf{Remark.} \textit{Try to do as much as possible and submit whatever progress you have. You can then look at the solutions after submitting. Try to spend atleast somewhere around 30-40 minutes on this set.}

\begin{enumerate}
    \item (MP4G 2022) Across the face of a rectangular post-it note, you idly draw lines that are parallel to its edges. Each time you draw a line, there is a $50\%$ chance it'll be in each direction and you never draw over an existing line or the edge of the post-it note. After a few minutes, you notice that you've drawn 20 lines. What is the expected number of rectangles that the post-it note will be partitioned into?
    \item (Folklore) Let $v_1,v_2,\ldots v_n\in \mathbb{R}^n$, all $|v_i|=1$ where $|x|$ refers to the Euclidean distance of $x$ from the origin.  Then there exist $\epsilon_1, \epsilon_2, \ldots \epsilon_n = \pm 1$ such that \[|\epsilon_1v_1+\epsilon_2v_2+\cdots \epsilon_nv_n|\le \sqrt{n}\] and also there exist $\epsilon_1, \epsilon_2, \ldots \epsilon_n = \pm 1$ such that \[|\epsilon_1v_1+\epsilon_2v_2+\cdots \epsilon_nv_n|\ge \sqrt{n}\] 
    \item (*) Suppose $p>n>10m^2$, with $p$ prime, and let $0<a_1<a_2<\cdots a_m<p$ be integers. Prove that there is an integer $0<x<p$ for which the $m$ numbers \[(xa_i \pmod{p})\pmod n\] are pairwise distinct.
\end{enumerate}

\section*{Solutions}

\subsection*{Problem 1}

\paragraph{\textbf{Problem 1. (MPFG 2022)}}Across the face of a rectangular post-it note, you idly draw lines that are parallel to its edges. Each time you draw a line, there is a $50\%$ chance it'll be in each direction and you never draw over an existing line or the edge of the post-it note. After a few minutes, you notice that you've drawn 20 lines. What is the expected number of rectangles that the post-it note will be partitioned into?

\begin{solution}
    Initially the paper is exactly one rectangle. Let $X_i$ be the RV corresponding to the number of more rectangles introduced by the $i$th line. Now, the final number of rectangles created is precisely $1+X_1+\cdots X_{20}$.\\
    
    Now, we have by linearity of expectation that \[\EE[1+X_1+\cdots X_{20}]=1+\EE[X_1]+\EE[X_2]+\cdots \EE[X_{20}]\]

    Now, $X_i$ just depends on the number of lines already drawn not parallel to it i.e. if $j$ lines are already drawn not parallel to it, it divides the rectangle into $j+1$ regions.\\

    Thus, $\EE[X_i]=1+\EE[\text{number of lines drawn not parallel to the $i$th line}]=1+\frac{i-1}{2}$ (each line is parallel to the $i$ th line with probability $\frac{1}{2}$).\\

    Thus, the final answer is $1+\frac{2}{2}+\frac{3}{2}+\cdots \frac{21}{2}=\frac{1+231}{2}=\boxed{116}$.\qed
\end{solution}


\subsection*{Problem 2}

\paragraph{\textbf{Problem 2. (Folklore)}} Let $v_1,v_2,\ldots v_n\in \mathbb{R}^n$, all $|v_i|=1$ where $|x|$ refers to the Euclidean distance of $x$ from the origin.  Then there exist $\epsilon_1, \epsilon_2, \ldots \epsilon_n = \pm 1$ such that \[|\epsilon_1v_1+\epsilon_2v_2+\cdots \epsilon_nv_n|\le \sqrt{n}\] and also there exist $\epsilon_1, \epsilon_2, \ldots \epsilon_n = \pm 1$ such that \[|\epsilon_1v_1+\epsilon_2v_2+\cdots \epsilon_nv_n|\ge \sqrt{n}\] 

\begin{solution}
    We let $v=\epsilon_1v_1+\epsilon_2v_2+\cdots \epsilon_nv_n$ where $v$ is a random variable and $\epsilon_i$ are picked as $\pm 1$ with probability $\frac{1}{2}$ each.\\

    Thus, finally let $X$ be the RV corresponding to $v\cdot v=|v|^2$ where $(\cdot)$ is used for the dot product.\footnote{check inequalities material if you don't remember definition of dot product.}.

    We will show that $\EE[X]=n$ which will immediately imply both the results! \[\EE[X]=\left(\sum_{i}\EE[\epsilon_i^2]v_i\cdot v_i\right)+\left(\sum_{i\ne j}\EE[\epsilon_i\epsilon_j]v_i\cdot v_j\right)\]
    \[\EE[\epsilon_i^2]=1\] since $\epsilon_i^2=1$ always and \[\EE[\epsilon_i\epsilon_j]=\EE[\epsilon_i]\EE[\epsilon_j]=0\cdot 0\]
    as $\epsilon_i$ and $\epsilon_j$ are independent. Finally, $v_i\cdot v_i=1$ for all $i$ as we know $\sqrt{v_i\cdot v_i}=|v_i|=1$ for all $i$. Thus, \[\EE[X]=\left(\sum_i {\bf 1}\right)+{\bf 0}=n\]
    Thus, we are done!
\end{solution}

\subsection*{Problem 3}

\paragraph{\textbf{Problem 3. (Probabilistic Method by Alon and Spencer Exercise. 2.7.4)}} Suppose $p>n>10m^2$, with $p$ prime, and let $0<a_1<a_2<\cdots a_m<p$ be integers. Prove that there is an integer $0<x<p$ for which the $m$ numbers \[(xa_i \pmod{p})\pmod n\] are pairwise distinct.

\begin{solution}
    Remind me to put up a solution. Writing it is more convoluted than I initially expected.
\end{solution}

\end{document}