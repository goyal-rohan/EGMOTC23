\documentclass[12pt]{article}

\usepackage{amsmath}
\usepackage{amssymb}
\usepackage{xcolor}
\usepackage[bmargin=1in]{geometry}
\usepackage[inline]{asymptote}
\usepackage{comment}

\title{Constructions PSet}
\author{EGMOTC 2023 - Rohan}
\date{\today}
\setlength{\parindent}{0pt}

\begin{document}

\maketitle

\newcommand{\localtextbulletone}{\textcolor{black}{\raisebox{.45ex}{\rule{.6ex}{.6ex}}}}
\renewcommand{\labelitemi}{\localtextbulletone}

\section*{Problems}
\vspace{1cm}
\thispagestyle{empty}

\textbf{Problem 1.}
    Determine whether for every real number $t$ such that $0 < t < \tfrac{1}{2} $ there exists an infinite set $S$ of positive integers such that\[|x-my| > ty\]for every pair of different elements $x$ and $y$ of $S$ and every positive integer $m$ (i.e. $m > 0$).\\


\textbf{Problem 2.} Prove that for every $n\in \mathbb N$, there exists a set $S$ of $n$ positive integers such that for any two distinct $a,b\in S$, $a-b$ divides $a$ and $b$ but none of the other elements of $S$.\\
%(USA TST 2015)

\textbf{Problem 3.} For which integers $n>1$ does there exist a rectangle that can be subdivided into $n$ pairwise noncongruent rectangles similar to the original rectangle?\\
%Miklos 2016

\textbf{Problem 4.} Euclid has a tool called cyclos which allows him to do the following:
\begin{itemize}
    \item Given three non-collinear marked points, draw the circle passing through them.
    \item Given two marked points, draw the circle with them as endpoints of a diameter.
    \item Mark any intersection points of two drawn circles or mark a new point on a drawn circle.
\end{itemize}
Show that given two marked points, Euclid can draw a circle centered at one of them and passing through the other, using only the cyclos.
%INMO 2023

\end{document}