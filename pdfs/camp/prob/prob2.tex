\documentclass[12pt]{article}

\usepackage{amsthm}
\usepackage{amsmath}
\usepackage{amssymb}
\usepackage{xcolor}
\usepackage[bmargin=1in]{geometry}
\usepackage[inline]{asymptote}
\usepackage{comment}
\newenvironment{solution}
{\paragraph{Solution.}}
{\qed\eject}
\usepackage{hyperref}
\newcommand{\EE}{\mathbb{E}}
\newcommand{\PP}{\mathbb{P}}
\newcommand{\NN}{\mathbb{N}}
\newcommand{\hyt}[2]{\hypertarget{#1 Claim #2}{\paragraph{Claim #2}}}
\newcommand{\hyl}[2]{Same as \hyperlink{#1 Claim #2}{Claim #2 in Solution #1}}
\newcommand{\hyr}[2]{\hyperlink{#1 Claim #2}{Claim #2}}
\DeclareMathOperator{\pow}{pow}
\newcommand{\nnt}{90^{\circ}}
\newcommand{\es}{\\[12pt]}
\usepackage{csquotes}
\usepackage{pythonhighlight}

\title{Probabilistic Method Lecture 2}
\author{EGMOTC 2023 - Rohan}
\date{\today}
\setlength{\parindent}{0pt}

\begin{document}

\maketitle

\newcommand{\localtextbulletone}{\textcolor{black}{\raisebox{.45ex}{\rule{.6ex}{.6ex}}}}
\renewcommand{\labelitemi}{\localtextbulletone}

\newtheorem{definition}{Definition}
\newtheorem{theorem}{Theorem}
\newtheorem{corollary}{Corollary}

\thispagestyle{empty}

\section*{Erd\"{o}s-Ko-Rado}

\paragraph{\textbf{Intersecting Set-families}}: A family $\mathcal{F}$ of sets is called intersecting if, $A,B\in \mathcal{F}\implies A\cap B\ne \Phi$. Now, suppose $n\ge 2k$ and let $\mathcal{F}$ be an intersecting family of $k$-element subsets of an $n$ set, say $[0,1,2,\cdots n-1]$. Then, Erd\"{o}s-Ko-Rado Theorem states that $|\mathcal{F}|\le \binom{n-1}{k-1}$. \footnote{Taken from the probabilistic lens 1 in Alon-Spencer.}

\section*{Maximal Antichains}

\paragraph*{Antichains:} A family $\mathcal{F}$ of subsets of $[n]$ is called an anti-chain if no set of $\mathcal{P}$ is contained in another. 

\begin{theorem}
    Let $\mathcal{F}$ be an antichain then, \[\sum_{A\in \mathcal{F}}\frac{1}{\binom{n}{|A|}}\le 1\]
\end{theorem}

The above theorem is also called the Lubell–Yamamoto–Meshalkin inequality or more frequently, the LYM inequality.

\begin{corollary} \textbf{(Sperner's Theorem)} Let $\mathcal{F}$ be an antichain, then 
    \[|\mathcal{F}|\le \binom{n}{\lfloor n/2 \rfloor}\]
\end{corollary}

Try to prove these three results.
\eject
\section*{Law of Large Numbers}

Take this result as something to remember intuitively and not necessarily too formally but more as an intuitive statement:

\paragraph*{\textbf{Law of Large Numbers.}} Let $X_1,\cdots X_n$ be indepedent identically distributed random values with expected value $0$. Then, if you define $\overline{X_n}=\frac{X_1+\cdots X_n}{n}$ then for any $\epsilon>0$, we have \[\lim_{n\mapsto \infty} \PP[|\overline{X_n}|<\epsilon]=1\]


\end{document}